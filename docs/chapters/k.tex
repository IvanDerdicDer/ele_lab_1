\chapter{Zadatak K} \label{ch:k}

U ovom poglavlju je opisan i odrađen zadatak K.

\section{Opis zadatka} \label{sec:k:opis}

na temelju podataka iz i) dijela zadatka i zadanog nazivnog napona baterijskog paketa i baterijske
ćelije, potrebno je na temelju zadane baterijske ćelije odrediti broj serijski i paralelno spojenih
baterijskih ćelija, izračunati nazivni napon, maksimalni napon punjenja, minimalni napon pražnjenja,
na temelju broja serijskih spojenih ćelija, te kapacitet baterijskog paketa, maksimalnu struju
punjenja, nazivnu struju punjenja, maksimalnu struju pražnjenja i nazivnu struju pražnjenjana
temelju broja paralelno spojenih ćelija.
Napomena k) : broj ćelija je potrebno zaokružiti na sljedeći veći cijeli broj, napone je potrebno
izračunati na temelju broja serijski spojenih ćelija, izračunati napon mora biti veći ili jednak od
zahtjevanog napona. Potrebni kapacitet izračunati na temelju ukupno utrošene energije iz h) dijela
zadatka i zadanog napona. Stvarni nazivni kapacitet je potrebno računati na temelju broja
paralelno spojenih ćelija, izračunati kapacitet mora biti veći ili jednak od potrebnog kapaciteta

\section{Rješenje} \label{sec:k:rjesenje}

Koristeći formule iz prezentacije, potrebno je izračunati broj ćelija, napon, kapacitet, struje punjenja i pražnjenja.

Dobivene vrijednosti su:
\begin{description}
    \item[broj ćelij u seriji] 154,
    \item[broj ćelij u paraleli] 282,
    \item[napon baterijskog paketa] 554.4 V,
    \item[maksimalni napon punjenja] 646.8 V,
    \item[maksimalni napon punjenja] 385 V,
    \item[kapacitet baterijskog paketa] 1410 Ah,
    \item[maksimalna struja punjenj] 1410 A,
    \item[nazivna struja punjenja] 705 A,
    \item[maksimalna struja pražnjenja] 4230 A i
    \item[nazivna struja pražnjenjana] 282 A.
\end{description}