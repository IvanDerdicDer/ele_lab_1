\chapter{Zadatak A} \label{ch:a}
U ovom poglavlju je opisan i odrađen zadatak A.

\section{Opis zadatka}
Prikazati krivulje kretanja u 2D i 3D obliku, krivulje prijeđenog puta [u SI jeidnicama], brzine vožnje[u
SI jeidnicama], ubrzanja[u SI jeidnicama], promjene visine[u SI jeidnicama], nagiba ceste [u
jedinicama : rad, °, \%]. Dodatno, za svaku varijablu je potrebno odrediti minimalnu, maksimalnu i
srednju vrijednost krivulje, te prikazati je u tabličnom obliku.
Napomena a) : potrebno je prikazati “čitljive“ podatke, to upućuje na korištenje filtera (kao
„movAvgFilt“) u svrhu izglađivanja krivulje, ali uz to da se ne izgubi stvarna slika krivulje. Potrebno
je filtrirati većim intezitetom i eventualno ograničit promjenu visine na način da nagib ceste ne
prelazi vrijednost od ±10\%.